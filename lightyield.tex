\chapter{Detector Design and Construction}\label{chapter:setup}
Two different scintillation-based detector-concepts were developed to measure the depth dose distribution of protons in real-time.
Both detectors consist of scintillator-layers measuring the energy deposition.

The first detector concept uses \ce{PbWO4} sheets with two thicknesses \SI{2}{\mm} and \SI{3}{\mm}.
Here the thicker \SI{3}{\mm} are used in the front to provide more stopping power and the thinner \SI{2}{\mm} are used in the Bragg-Peak region to give the optimal spatial resolution whilst still only utilizing 32 channels.
The thicknesses were chosen to fully cover the \SI{66}{\mm} range of \SI{220}{\mega\electronvolt} protons in \ce{PbWO4}, yet still being thick enough not to break as \ce{PbWO4} is a very fragile material.

The second detector concept uses EJ-212 plastic scintillators with a thickness of \SI{4}{\mm} allowing for a better spatial resolution.
However, since the 32 available channels with \SI{4}{\mm} Ej-212 only cover $\approx \SI{12.8}{\centi\meter}$ of the \SI{30.4}{\centi\meter} proton range, a passive absorber has to be included to fully stopp the protons inside the detector.
This \SI{20}{\centi\meter} passive absorber consists of PMMA resulting in enough stopping power to stop the beam inside the detector.
The passive absorber is placed between the first and second plastic scintillator such that the first works as a trigger channel which is used for the normalization in the analysis.

\section{Readout}\label{section:readout}
The scintillators of both designs are read out via SiPMs.
Light yield measurements were conducted, to decide which SiPM types are suitable.
With these the amount of incident photons can be estimated and compared with the number of pixel.
From this, a balance can be struck between high resolution and a large enough dynamic range.
 
\subsection{Light Yield Measurement}
The measurements were conducted using the process described in Section~\ref{section:lightyield} and the setup shown in Figure~\ref{fig:lightyield:setup}.
The PMT used is an R2059 from Hamamatsu (serial number BA3200) with a quantum efficiency of \SI{23.16}{\percent}~\cite{datasheet:hamamatsu_R2059} (cf. Appendix~\ref{appendix:spectral}) at the luminescence peak of \SI{420}{\nano\meter} of \ce{PbWO4}~\cite{cms:tdr} and EJ212.

\subsubsection{Light Yield: \ce{PbWO4}}
The \ce{PbWO4} measurement were done in a flat and vertical position as shown in Figure~\ref{fig:detector:lightyield:pwo}, where all non PMT-facing scintillator sides were enveloped in highly reflective PTFE foil in order to not lose any photons.
Two additional measurements were performed, where one \SI{3}{\milli\meter}- and \SI{2}{\milli\meter} crystal were fully wrapped with an SiPM sized window cutout in the center of one side  as shown in Figure~\ref{fig:detector:crystal:pwo:wrap}.
The \ce{PbWO4} crystals were mounted onto the PMT's optical window next to a \ce{^{22}Na} $\gamma$-source inside a climate chamber.
The optical coupling was done using glycerol ($n=1.4722$), as shown in Figure~\ref{fig:pwo:lightyield:open}.
Glycerin was used as a substitute for the commonly used Baysilone\textsuperscript{{\textregistered}} Fluid~M optical grease ($n\approx 1.404,~\eta=\SI{300000}{\milli\meter^2\per\second}$~\cite{bayer:baysilone}), due to its less-adhesive characteristic.
The Baysilone\textsuperscript{{\textregistered}} Fluid~M with its high adhesion might have lead to damaging the fragile crystals during removal.
The refractive index of \ce{PbWO4} and the \ce{SiO2} glass window of the \gls{PMT} are $n_{\ce{PbWO4}}\approx2.3$~\cite{cms:tdr} and $n_{\ce{SiO2}}\approx1.459$~\cite{Malitson:65}, respectively.
Additionally to the climate chamber's light-tightness, the setup is enclosed in PTFE foil, ensuring perfect light tightness, as shown in Figure~\ref{fig:detector:lightyield:setup:closed}.

\begin{figure}[h!]
    \centering
    \begin{minipage}[t]{.5\textwidth}
        \centering
        \captionsetup{width=.9\linewidth}
        %\includegraphics[width=.8\linewidth]{}
        \captionof{figure}{Open light yield measurement setup for \ce{PbWO4} crystals using a \ce{^{22}Na} $\gamma$-source.}\label{fig:pwo:lightyield:open}
    \end{minipage}%
    \begin{minipage}[t]{.5\textwidth}
        \centering
        \captionsetup{width=.9\linewidth}
        %\includegraphics[width=.8\linewidth]{}
        \captionof{figure}{Encased light yield measurement setup for \ce{PbWO4} crystals using a \ce{^{22}Na} $\gamma$-source.}\label{fig:pwo:lightyield:closed}
    \end{minipage}
\end{figure}

The measurements were conducted at \SI{20}{\celsius} for \SI{5}{\minute} after an acclimation time of \SI{5}{\minute} each.
The acclimation time was chosen small because the crystals were keept inside the climate chamber for \SI{24}{\hour} before the measurements were startet, thereby only the short time frame inbetween measuements, where the chamber was opened, had to be accounted for.
An exemplary light yield measurement of crystal number~0 in the flat position is shown in Figure~\ref{fig:detector:lightyield:pwo:measurement:flat:0}.
The measured light yield values of all crystals for the different setups are shown in Figure~\ref{}.
The \SI{3}{\milli\meter} thick crystals average approximately \SI{164.73}{ph\per\mega\electronvolt} and \SI{129.44}{ph\per\mega\electronvolt} in the flat and vertical positions, respectively.
The \SI{2}{\milli\meter} thick crystals average approximately \SI{131.00}{ph\per\mega\electronvolt} and \SI{94.32}{ph\per\mega\electronvolt} in the flat and vertical positions, respectively.
With the SiPM-sized window cutout the light yield of a \SI{3}{\milli\meter} and \SI{2}{\milli\meter} crystal was \SI{75.25\pm 33.09}{ph\per\mega\electronvolt} and \SI{59.7\pm 19.22}{ph\per\mega\electronvolt}, respectively.

\subsubsection{Light Yield: \ce{EJ-200}}
The light yield of a $50 \times 50 \times 10 \si{\milli\meter\cubed}$ EJ-200 sample was measured to estimate the amount of incident photons on an SiPM by a plastic scintillator to decide which SiPM type is needed for the readout.

  
% \begin{figure}[h!]
%     \centering
%     \includegraphics[width=0.55\linewidth]{braggsampler/fig/lightyield/lightyieldmeasurement_ch0.pdf}
%     \caption{Exemplary light yield measurement of crystal number 0 with Gaus fitted \ce{^{137}Cs} source peak.}\label{fig:pwo:lightyield:measurement}
% \end{figure}

% \begin{figure}
%     \centering
%     \includegraphics[width=0.55\linewidth]{braggsampler/fig/lightyield/lightyield_plot.pdf}
%     \caption{Plotted light yield measurements of the Braggpeak Sampler's crystals.}\label{fig:pwo:lightyield:plot}
% \end{figure}