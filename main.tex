\documentclass[a4paper,12pt,twoside]{report}
\usepackage{graphicx}
\usepackage{amsmath}
\usepackage{float}
\usepackage{subcaption}

% Mehr mathematische Symbole
\usepackage{amssymb}

%Braket-Notation
\usepackage{braket}

\raggedbottom{}

%Griechische Buchstaben, nicht kursiv
\usepackage{upgreek}
%Bibliography
\usepackage[style=numeric,sorting=none,backend=biber]{biblatex}
\addbibresource{bibliography.bib}
% \DefineBibliographyStrings{german}{andothers = {et\addabbrvspace{} al\adddot}}

%Zeilenabstand
\usepackage[onehalfspacing]{setspace}

%Seitenränder
\usepackage[a4paper, left=3cm, right=3cm]{geometry}

%Chemie
\usepackage[version=4]{mhchem}

%Header
\usepackage{fancyhdr}
\pagestyle{fancy}
\setlength{\headheight}{15pt}
\newcommand{\myol}[2][3]{{}\mkern#1mu\overline{\mkern-#1mu#2}}

%Schriftarten
\usepackage{mathptmx}
\usepackage[T1]{fontenc}
\usepackage[scaled]{helvet}

%Formatierung Überschriften
\usepackage{titlesec}
\titleformat{\chapter}[display]
  {\normalfont\sffamily\Huge\bfseries}
  {\chaptertitlename\ \thechapter}{10pt}{\Huge}
\titleformat{\section}[hang]
    {\normalfont\sffamily\Large\bfseries}
    {\thesection}{10pt}{\Large}
\titleformat{\subsection}[hang]
    {\normalfont\sffamily\Large\bfseries}
    {\thesubsection}{10pt}{\Large}

%Keine Einzüge bei neuem Absatz
\setlength{\parindent}{0pt}
\setlength{\parskip}{5pt plus 2pt minus 1pt}

%Syntax Highlighting
% \usepackage[most, minted]{tcolorbox}
% \tcbuselibrary{minted,breakable}
% \usepackage{xcolor}
% \usemintedstyle{perldoc}
% \definecolor{bg}{rgb}{0.95,0.95,0.95}
% \renewcommand{\theFancyVerbLine}{\scriptsize {\arabic{FancyVerbLine}}}
% \newcommand{\myminted}[1]{\tcbinputlisting{
% colback=bg,
% breakable,
% listing only,
% listing file={#1},
% minted language=c++,
% minted options={linenos,breaklines,numbersep=2mm,xleftmargin=2mm},
% enhanced,
% overlay={%
%        \begin{tcbclipinterior}
%            \fill[gray!25] (frame.south west) rectangle ([xshift=6mm]frame.north west);
%        \end{tcbclipinterior}
%    }
% }}

%Lorem Ipsum
\usepackage{lipsum}

%Flowcharts
\usepackage{tikz}
\usetikzlibrary{shapes.geometric, arrows,shapes.symbols}
\tikzstyle{startstop} = [rectangle, rounded corners, minimum width=3cm, minimum height=1cm,text centered, draw=black]
\tikzstyle{process} = [rectangle, minimum width=3cm, minimum height=1cm, text width=4cm, text centered, draw=black]
\tikzstyle{decision} = [signal,signal to=east and west, minimum width=3cm, minimum height=1cm, text centered, text width=3cm,draw=black]
\tikzstyle{object} = [rectangle, minimum height=1cm, text centered, draw=black]
\tikzstyle{arrow} = [thick,->,>=stealth]
\tikzstyle{line} = [thick,-,>=stealth]

%Literatur-, Abbildungs- uend Tabellenverzeichnis in Inhaltsverzeichnis
\usepackage[nottoc, numbib]{tocbibind}

% Subsubsections mit Zahlen
\setcounter{secnumdepth}{3}
\setcounter{tocdepth}{3}

%Metadaten und Verlinkungen
\usepackage{hyperref}
\hypersetup{
  pdftitle={PhD Thesis},
  pdfauthor={Niclas Fiedler},
  breaklinks=true,
  }

%URLs mit line break
\usepackage{xurl}

%Einheiten
\usepackage[separate-uncertainty=true,per-mode=symbol]{siunitx}
\usepackage{physics}
\ExplSyntaxOn{}
\msg_redirect_name:nnn { siunitx } { physics-pkg } { none }
\ExplSyntaxOff{}
\DeclareSIUnit{} \clight{\text{\ensuremath{c}}}
\DeclareSIUnit{} \u{\text{u}}
\sisetup{per-mode=fraction}
\DeclareSIUnit\ppm{\text{ppm}}
\DeclareSIUnit\clight{\text{c}}
\DeclareSIUnit\lsb{\text{LSB}}
\DeclareSIUnit\pe{\text{PE}}
\DeclareSIUnit\bins{\text{bins}}
\DeclareSIUnit[]\voltpp{\text{\ensuremath{V_{\textup{pp}}}}}
\DeclareSIUnit\atomicmassunit{u}
\DeclareSIUnit\barn{b}

%PDFs einfügen
\usepackage{pdfpages}

\usepackage{adjustbox}

%Akronyme
\usepackage[acronym, shortcuts, toc, nopostdot]{glossaries}[=v4.46]
\usepackage{glossary-mcols}
\glssetwidest{MTCoMPASS}
\makeatletter
  \newglossarystyle{mymcolalttree}{%
    \setglossarystyle{alttree}%
    \renewenvironment{theglossary}%
      {%
        \setlength{\columnsep}{23pt}
        \begin{multicols}{2}%
        \def\@gls@prevlevel{-1}%
        \mbox{}\par
        \vspace*{-1\baselineskip}
      }%
      {\par\end{multicols}}%
    \renewcommand{\glossaryentryfield}[5]{%
      \ifnum\@gls@prevlevel=0
      \hangindent\glstreeindent{}
      \else
        \settowidth{\glstreeindent}{\textbf{\@glswidestname\space}}%
        \hangindent\glstreeindent{}
        \parindent\glstreeindent{}
      \fi
      \makebox[0pt][r]{\makebox[\glstreeindent][l]{%
        \glsentryitem{##1}\textbf{\glstarget{##1}{##2}}}}%
      \ifx\relax##4\relax
      \else
        (##4)\space
      \fi
      ##3\glspostdescription{} \space ##5\par
      \def\@gls@prevlevel{0}%
    }%
  }%
\makeatother
  \renewcommand{\glsgroupskip}{}
\makeglossaries{}
\input{abbreviations.txt}

%Plots
\usepackage{pgfplots}
\pgfplotsset{compat=newest}
\pgfplotsset{
  layers/my layer set/.define layer set={
      background,
      main,
      foreground
  }{
  },
  set layers=my layer set,
}

% Multidigit float counters
\usepackage{alphalph}
\renewcommand*{\thesubfigure}{\alphalph{\value{subfigure}}}

%Float barrier
\usepackage{placeins}

% Multipage Tables
\usepackage{longtable}

% Zeilenumbruch in Tabelle
\usepackage{makecell}

% Dynamic width tables
\usepackage{ltxtable}

%Mehrere Rows in Tabelle
\usepackage{multirow}

% %Deutsche Sprache
\usepackage[ngerman, english]{babel}

% %Bibliography mit Babel
\usepackage{csquotes}

%Reference auf Text
\makeatletter
\newcommand\newtag[2]{#1\def\@currentlabel{#1}\label{#2}}
\makeatother

%Kein Reset von Page number
\usepackage{etoolbox}
\makeatletter
\patchcmd{\titlepage}{\setcounter{page}\@ne}{}     {}{\PackageError{titlepage}{failed to patch \string\begin{titlepage} to not reset the page counter}{}}
\patchcmd{\endtitlepage}{\setcounter{page}\@ne}{}  {}{\PackageError{titlepage}{failed to patch \string\end{titlepage} to not reset the page counter}{}}
\makeatother

\begin{document}

\pagenumbering{roman}
\begin{titlepage}
   \begin{center} 
      \vspace*{1cm}
      \includegraphics[scale=1.2]{fig/jlu.png}
      \vspace{1.5cm}

      \textbf{\LARGE{PhD Thesis}} % 
  
      \vspace{1.5cm}
  
      {\large{PhD Thesis}}
  
      \vspace{1.5cm}
  
      PhD Thesis\\
      \textbf{by Niclas Lars Rudolf Fiedler, M.Sc.}\\
      from Weilmünster\\
      
      XX.XX.2028
      \vfill
      Justus Liebig University Gießen\\
      Institute of Experimental Physics II
      \vspace{0.8cm}
      
      First Examiner: Prof.\ Dr.\ Kai-Thomas Brinkmann\\
      Second Examiner: Prof.\ Dr.\ Jens Sören Lange, AkR\\
      Supervisor: Dr.~Dzmitry Kazlou 
   \end{center}
\end{titlepage}

\newpage\null\thispagestyle{empty}\newpage

% Auszug
\selectlanguage{english}
\phantomsection{}
\addcontentsline{toc}{chapter}{Abstract}
\begin{abstract}
  \thispagestyle{plain}
  
\end{abstract}

\selectlanguage{ngerman}
\phantomsection{}
\addcontentsline{toc}{chapter}{Zusammenfassung}
\begin{abstract}
  \thispagestyle{plain}
  
\end{abstract}
\selectlanguage{english}

\tableofcontents
\clearpage
\pagenumbering{arabic}

\chapter{Detector Design and Construction}\label{chapter:setup}
Two different scintillation-based detector-concepts were developed to measure the depth dose distribution of protons in real-time.
Both detectors consist of scintillator-layers measuring the energy deposition.

The first detector concept uses \ce{PbWO4} sheets with two thicknesses \SI{2}{\mm} and \SI{3}{\mm}.
Here the thicker \SI{3}{\mm} are used in the front to provide more stopping power and the thinner \SI{2}{\mm} are used in the Bragg-Peak region to give the optimal spatial resolution whilst still only utilizing 32 channels.
The thicknesses were chosen to fully cover the \SI{66}{\mm} range of \SI{220}{\mega\electronvolt} protons in \ce{PbWO4}, yet still being thick enough not to break as \ce{PbWO4} is a very fragile material.

The second detector concept uses EJ-212 plastic scintillators with a thickness of \SI{4}{\mm} allowing for a better spatial resolution.
However, since the 32 available channels with \SI{4}{\mm} Ej-212 only cover $\approx \SI{12.8}{\centi\meter}$ of the \SI{30.4}{\centi\meter} proton range, a passive absorber has to be included to fully stopp the protons inside the detector.
This \SI{20}{\centi\meter} passive absorber consists of PMMA resulting in enough stopping power to stop the beam inside the detector.
The passive absorber is placed between the first and second plastic scintillator such that the first works as a trigger channel which is used for the normalization in the analysis.

\section{Readout}\label{section:readout}
The scintillators of both designs are read out via SiPMs.
Light yield measurements were conducted, to decide which SiPM types are suitable.
With these the amount of incident photons can be estimated and compared with the number of pixel.
From this, a balance can be struck between high resolution and a large enough dynamic range.
 
\subsection{Light Yield Measurement}
The measurements were conducted using the process described in Section~\ref{section:lightyield} and the setup shown in Figure~\ref{fig:lightyield:setup}.
The PMT used is an R2059 from Hamamatsu (serial number BA3200) with a quantum efficiency of \SI{23.16}{\percent}~\cite{datasheet:hamamatsu_R2059} (cf. Appendix~\ref{appendix:spectral}) at the luminescence peak of \SI{420}{\nano\meter} of \ce{PbWO4}~\cite{cms:tdr} and EJ212.

\subsubsection{Light Yield: \ce{PbWO4}}
The \ce{PbWO4} measurement were done in a flat and vertical position as shown in Figure~\ref{fig:detector:lightyield:pwo}, where all non PMT-facing scintillator sides were enveloped in highly reflective PTFE foil in order to not lose any photons.
Two additional measurements were performed, where one \SI{3}{\milli\meter}- and \SI{2}{\milli\meter} crystal were fully wrapped with an SiPM sized window cutout in the center of one side  as shown in Figure~\ref{fig:detector:crystal:pwo:wrap}.
The \ce{PbWO4} crystals were mounted onto the PMT's optical window next to a \ce{^{22}Na} $\gamma$-source inside a climate chamber.
The optical coupling was done using glycerol ($n=1.4722$), as shown in Figure~\ref{fig:pwo:lightyield:open}.
Glycerin was used as a substitute for the commonly used Baysilone\textsuperscript{{\textregistered}} Fluid~M optical grease ($n\approx 1.404,~\eta=\SI{300000}{\milli\meter^2\per\second}$~\cite{bayer:baysilone}), due to its less-adhesive characteristic.
The Baysilone\textsuperscript{{\textregistered}} Fluid~M with its high adhesion might have lead to damaging the fragile crystals during removal.
The refractive index of \ce{PbWO4} and the \ce{SiO2} glass window of the \gls{PMT} are $n_{\ce{PbWO4}}\approx2.3$~\cite{cms:tdr} and $n_{\ce{SiO2}}\approx1.459$~\cite{Malitson:65}, respectively.
Additionally to the climate chamber's light-tightness, the setup is enclosed in PTFE foil, ensuring perfect light tightness, as shown in Figure~\ref{fig:detector:lightyield:setup:closed}.

\begin{figure}[h!]
    \centering
    \begin{minipage}[t]{.5\textwidth}
        \centering
        \captionsetup{width=.9\linewidth}
        %\includegraphics[width=.8\linewidth]{}
        \captionof{figure}{Open light yield measurement setup for \ce{PbWO4} crystals using a \ce{^{22}Na} $\gamma$-source.}\label{fig:pwo:lightyield:open}
    \end{minipage}%
    \begin{minipage}[t]{.5\textwidth}
        \centering
        \captionsetup{width=.9\linewidth}
        %\includegraphics[width=.8\linewidth]{}
        \captionof{figure}{Encased light yield measurement setup for \ce{PbWO4} crystals using a \ce{^{22}Na} $\gamma$-source.}\label{fig:pwo:lightyield:closed}
    \end{minipage}
\end{figure}

The measurements were conducted at \SI{20}{\celsius} for \SI{5}{\minute} after an acclimation time of \SI{5}{\minute} each.
The acclimation time was chosen small because the crystals were keept inside the climate chamber for \SI{24}{\hour} before the measurements were startet, thereby only the short time frame inbetween measuements, where the chamber was opened, had to be accounted for.
An exemplary light yield measurement of crystal number~0 in the flat position is shown in Figure~\ref{fig:detector:lightyield:pwo:measurement:flat:0}.
The measured light yield values of all crystals for the different setups are shown in Figure~\ref{}.
The \SI{3}{\milli\meter} thick crystals average approximately \SI{164.73}{ph\per\mega\electronvolt} and \SI{129.44}{ph\per\mega\electronvolt} in the flat and vertical positions, respectively.
The \SI{2}{\milli\meter} thick crystals average approximately \SI{131.00}{ph\per\mega\electronvolt} and \SI{94.32}{ph\per\mega\electronvolt} in the flat and vertical positions, respectively.
With the SiPM-sized window cutout the light yield of a \SI{3}{\milli\meter} and \SI{2}{\milli\meter} crystal was \SI{75.25\pm 33.09}{ph\per\mega\electronvolt} and \SI{59.7\pm 19.22}{ph\per\mega\electronvolt}, respectively.

\subsubsection{Light Yield: \ce{EJ-200}}
The light yield of a $50 \times 50 \times 10 \si{\milli\meter\cubed}$ EJ-200 sample was measured to estimate the amount of incident photons on an SiPM by a plastic scintillator to decide which SiPM type is needed for the readout.

  
% \begin{figure}[h!]
%     \centering
%     \includegraphics[width=0.55\linewidth]{braggsampler/fig/lightyield/lightyieldmeasurement_ch0.pdf}
%     \caption{Exemplary light yield measurement of crystal number 0 with Gaus fitted \ce{^{137}Cs} source peak.}\label{fig:pwo:lightyield:measurement}
% \end{figure}

% \begin{figure}
%     \centering
%     \includegraphics[width=0.55\linewidth]{braggsampler/fig/lightyield/lightyield_plot.pdf}
%     \caption{Plotted light yield measurements of the Braggpeak Sampler's crystals.}\label{fig:pwo:lightyield:plot}
% \end{figure}
\subsubsection{Depth Calculation}
In particle therapy, the measurement depth $z$ in a detector is conventionally expressed as the water-equivalent thickness~(WET).
The WET $t$ of a material layer is the thickness of water that causes the same energy loss for a proton beam.
For this, the physical depth of the detector must be converted to a WET, allowing for direct comparisons.

The conversion is calculated by approximating the stopping power ratio of \ce{PbWO4} and \ce{H2O}, through the energy-range relation~\citep{bortfeld} at the nominal beam energy of \SI{221.6}{\mega\electronvolt}.
\begin{equation}
  t = \frac{S_{\ce{PbWO4}}}{S_{\ce{H2O}}}\cdot x = \frac{\frac{\mathrm{d}E_{\ce{PbWO4}}}{\mathrm{d}x}}{\frac{\mathrm{d}E_{\ce{H2O}}}{\mathrm{d}x}} \cdot x = \frac{\frac{1}{\alpha_{\ce{PbWO4}}\cdot p_{\ce{PbWO4}}}E^{1-p_{\ce{PbWO4}}}}{\frac{1}{\alpha_{\ce{H2O}}\cdot p_{\ce{H2O}}}E^{1-p_{\ce{H2O}}}}\cdot x= \frac{\alpha_{\ce{H2O}}\cdot p_{\ce{H2O}}E^{1-p_{\ce{PbWO4}}}}{\alpha_{\ce{PbWO4}}\cdot p_{\ce{PbWO4}} E^{1-p_{\ce{H2O}}}}\cdot x
\end{equation}


\#\#\#\#\# Anfang alternativerer komplexerer Gleichung (braucht vielleicht noch Simulationsvalidierung, vielleicht was für ein anderes Paper)

The conversion is calculated iteratively for each material layer of the detector (e.g. wrapping, crystal), starting from the beam entrance.
By definition, a proton's energy after traversing a depth $t$ in water is identical to its energy after traversing a depth $x$ in another material, where $t$ is the WET corresponding to $x$.
Using the empirical range-energy relationship for protons in matter~\citep{bortfeld}, the conversion of a materials depth $x$ to WET $t$ can be expressed as

\begin{align}
    E(t)_{\ce{H2O}} &= E(x)_{\text{mat}} \\
    \left(\frac{t}{\alpha_{\ce{H2O}}}\right)^{\frac{1}{p_{\ce{H2O}}}} &= \left(\frac{x}{\alpha_{\text{mat}}}\right)^{\frac{1}{p_{\text{mat}}}} \\
    t&= \alpha_{\ce{H2O}}\left(\frac{x}{\alpha_{\text{mat}}}\right)^{\frac{p_{\ce{H2O}}}{p_{\text{mat}}}}, \label{eq:wet:conversion:single}
\end{align}
where $\alpha$ and $p$ are the material-dependent parameters of the range--energy relation.
In this formulation, $p$ is assumed to be largely energy independent.

To extend this formalism to a detector consisting of multiple layers of different materials, an iterative approach is used.
The cumulative WET after $n$ layers, $t_n$, is calculated from the WET after $n-1$ layers, $t_{n-1}$, and the physical thickness of the current layer $\Delta x$.
The corresponding physical depth after $n$ layers is given by
\begin{equation}
  \label{eq:wet:conversion:physical}
    x_n = x_{n-1} + \Delta x .
\end{equation}
By inserting $x_{n}$ into $t_{n}$ of Equation~\ref{eq:wet:conversion:physical}, and expressing $x_{n-1}$ in terms of $t_{n-1}$, the WET can be calculated iteratively as

\begin{align}
  t_n &= \alpha_{\ce{H2O}}\left(\frac{x_{n-1}+\Delta x}{\alpha_{\text{mat}}}\right)^{\frac{p_{\ce{H2O}}}{p_{\text{mat}}}} \\
  t_n &= \alpha_{\ce{H2O}} \left[ \left( \frac{t_{n-1}}{\alpha_{\ce{H2O}}} \right)^{\frac{p_{\text{mat}}}{p_{\ce{H2O}}}} + \frac{\Delta x}{\alpha_{\text{mat}}} \right]^{\frac{p_{\ce{H2O}}}{p_{\text{mat}}}},\label{eq:wet:conversion:multi}
\end{align}

with the boundary conditions $t_0 = 0$ and $x_0 = 0$ at the detector entrance.

\#\#\#\#\# Ende von iterativer Gleichung

Assuming a homogeneous interaction probability inside each scintillator layer, the effective measurement depth $z_n$ of layer $n$ is taken as the midpoint of the WET interval,

\begin{equation}
 z_n = \frac{t_{n-1}+t_n}{2}.
\end{equation}

Under the same assumption of a uniform distribution, the uncertainty of the measurement depth is given by
\begin{equation}
    \sigma_{z_n}^2 =
    \frac{1}{12}\left(t_n - t_{n-1}\right)^2.
\end{equation}  

\paragraph{Material Parameters Determination}

\#\#\# Note: Falls einfachere WET Berechnung, hier nur Werte für PbWO4 herleiten

The material-specific parameters $\alpha$ and $p$ for water and the detector components (\ce{PbWO4}, PTFE and aluminum) are used in the conversion of the detector depth to WET.
They can be determined empirically by performing an exponential fit to proton range-energy data~\citep{bortfeld}.
For liquid water and aluminum, the range-energy reference data were obtained directly from~\citet{ICRU49}.
Measured range-energy data for \ce{PbWO4} and PTFE were not available.
Instead, the corresponding range-energy tables were generated using a Geant4 (11.2.1) simulation.
For that the \texttt{QGSP\_BIC\_EMY} physics list was used with the electromagnetic physics constructor substituted by \texttt{G4EmStandardPhysics\_option4}.
The PTFE membrane is approximated by its core chemical composition of \ce{C2F4} chains with a density of \SI{0.35}{\g\per\cm\cubed} as provided by the vendor.
The inbuilt \texttt{G4\_PbWO4} was used for the \ce{PbWO4} material definition with a density of \SI{8.28}{\g\per\cm\cubed}.
The accuracy of the simulation was validated by comparing the calculated proton ranges for liquid water against the reference data from~\citet{ICRU49}.

The simulated ranges deviated about \SI{1}{\percent} for small energies with the deviation decreasing for increasing energies to about \SI{0.5}{\percent}, indicating a good accuracy of the simulation.
The resulting range-energy data for \ce{PbWO4} and PTFE were subsequently fitted and their respective $\alpha$ and $p$ extracted.
The final parameters of all used materials are shown in Table~\ref{tab:rangeenergy:fit} with the corresponding fits illustrated in Figure~\ref{fig:rangeenergy:fit}.
\begin{figure}[htb]
  \centering
  \includegraphics[width=0.5\textwidth]{figs/rangeenergy.pdf}
  \caption{Exponential fits using the range-energy relation~\citep{bortfeld} to range-energy data for protons in liquid \ce{H2O} and solid \ce{Al} from the~\citet{ICRU49} and for \ce{PbWO4} and PTFE from a Geant4 simulation.}
  \label{fig:rangeenergy:fit}  
\end{figure}

\begin{table}[b]
    \centering
    \captionof{table}{Fitted material parameters $\alpha$ and $p$ using the range-energy relation~\citep{bortfeld} and the range-energy data from~\citet{ICRU49} and a Geant4 simulation.}\label{tab:rangeenergy:fit}
    \begin{tabular}{lccc}
    \toprule
    Material & $\alpha$ / $\frac{\text{mm}}{\text{MeV}^p}$& $p$ & $\mathrm{Cov}(\alpha, p)$\\
    \midrule
    \ce{PbWO4} & $7.275\times 10^{-3}\pm 1.497\times 10^{-4}$ & $1.69\pm 3.865\times 10^{-3}$ & $-5.781\times 10^{-7}$\\
    PTFE & $9.158\times 10^{-2}\pm 2.387\times 10^{-3}$ & $1.734\pm 4.890\times 10^{-3}$ & $-1.166\times 10^{-5}$\\
    Water & $2.585\times 10^{-2}\pm 6.826\times 10^{-4}$ & $1.738\pm 4.953\times 10^{-3}$ & $-3.377\times 10^{-6}$\\
    Aluminum & $1.319\times 10^{-2}\pm 3.224\times 10^{-4}$ & $1.725\pm 4.588\times 10^{-3}$ & $-1.477\times 10^{-6}$\\
    \bottomrule
    \end{tabular}
\end{table}


\section{Range-Energy Simulation Validation}\label{appendix:rangeenergyvalidation}

\begin{table}[htbp]
    \centering
    \caption{Measured proton CSDA range~\citet{ICRU49} and simulated range in \ce{H2O}, \ce{PbWO4}, PTFE and aluminum for various energies. The deviation between measured and simulated range in \ce{H2O} is also given.}
    \def\arraystretch{1.1}
    \begin{tabular}{c|c|c|c||c|c|c}
      \makecell{Energy \\\relax [\si{\mega\electronvolt}]} & \makecell{ICRU range \\\relax in \ce{H2O} [\si{\milli\meter}]} & \makecell{Simulated range \\\relax in \ce{H2O} [\si{\milli\meter}]} & \makecell{Deviation \\\relax [\si{\percent}]} & \makecell{ICRU range \\\relax in \ce{Al} [\si{\milli\meter}]} & \makecell{Simulated range \\\relax in \ce{PbWO4} [\si{\milli\meter}]} & \makecell{Simulated range \\\relax in \ce{PTFE} [\si{\milli\meter}]} \\
      \hline
        $3$ & \makecell{$\num{1.50e-01} \pm $\\$ \num{2.25e-03}$} & \makecell{$\num{1.51e-01} \pm $\\$ \num{2.28e-03}$} & $0.589$ & \makecell{$\num{8.13e-02} \pm $\\$ \num{1.22e-03}$} & \makecell{$\num{5.08e-02} \pm $\\$ \num{1.05e-03}$} & \makecell{$\num{8.78e-02} \pm $\\$ \num{2.49e-04}$} \\
        \hline
        $5$ & \makecell{$\num{3.62e-01} \pm $\\$ \num{5.43e-03}$} & \makecell{$\num{3.65e-01} \pm $\\$ \num{4.64e-03}$} & $0.636$ & \makecell{$\num{1.91e-01} \pm $\\$ \num{2.87e-03}$} & \makecell{$\num{1.13e-01} \pm $\\$ \num{1.16e-03}$} & \makecell{$\num{2.09e-01} \pm $\\$ \num{1.93e-03}$} \\
        \hline
        $10$ & \makecell{$\num{1.23e+00} \pm $\\$ \num{1.84e-02}$} & \makecell{$\num{1.24e+00} \pm $\\$ \num{1.61e-02}$} & $0.714$ & \makecell{$\num{6.29e-01} \pm $\\$ \num{9.43e-03}$} & \makecell{$\num{3.42e-01} \pm $\\$ \num{4.92e-03}$} & \makecell{$\num{6.98e-01} \pm $\\$ \num{8.07e-03}$} \\
        \hline
        $15$ & \makecell{$\num{2.54e+00} \pm $\\$ \num{3.81e-02}$} & \makecell{$\num{2.56e+00} \pm $\\$ \num{3.27e-02}$} & $0.750$ & \makecell{$\num{1.28e+00} \pm $\\$ \num{1.92e-02}$} & \makecell{$\num{6.68e-01} \pm $\\$ \num{9.74e-03}$} & \makecell{$\num{1.43e+00} \pm $\\$ \num{1.70e-02}$} \\
        \hline
        $20$ & \makecell{$\num{4.26e+00} \pm $\\$ \num{6.39e-02}$} & \makecell{$\num{4.30e+00} \pm $\\$ \num{5.28e-02}$} & $1.027$ & \makecell{$\num{2.12e+00} \pm $\\$ \num{3.18e-02}$} & \makecell{$\num{1.09e+00} \pm $\\$ \num{1.57e-02}$} & \makecell{$\num{2.39e+00} \pm $\\$ \num{2.86e-02}$} \\
        \hline
        $30$ & \makecell{$\num{8.85e+00} \pm $\\$ \num{1.33e-01}$} & \makecell{$\num{8.95e+00} \pm $\\$ \num{1.05e-01}$} & $1.047$ & \makecell{$\num{4.35e+00} \pm $\\$ \num{6.53e-02}$} & \makecell{$\num{2.17e+00} \pm $\\$ \num{3.08e-02}$} & \makecell{$\num{4.94e+00} \pm $\\$ \num{5.85e-02}$} \\
        \hline
        $40$ & \makecell{$\num{1.49e+01} \pm $\\$ \num{2.23e-01}$} & \makecell{$\num{1.50e+01} \pm $\\$ \num{1.74e-01}$} & $0.877$ & \makecell{$\num{7.27e+00} \pm $\\$ \num{1.09e-01}$} & \makecell{$\num{3.56e+00} \pm $\\$ \num{4.68e-02}$} & \makecell{$\num{8.27e+00} \pm $\\$ \num{9.71e-02}$} \\
        \hline
        $50$ & \makecell{$\num{2.23e+01} \pm $\\$ \num{3.34e-01}$} & \makecell{$\num{2.24e+01} \pm $\\$ \num{2.52e-01}$} & $0.806$ & \makecell{$\num{1.08e+01} \pm $\\$ \num{1.62e-01}$} & \makecell{$\num{5.26e+00} \pm $\\$ \num{6.51e-02}$} & \makecell{$\num{1.24e+01} \pm $\\$ \num{1.44e-01}$} \\
        \hline
        $60$ & \makecell{$\num{3.09e+01} \pm $\\$ \num{4.64e-01}$} & \makecell{$\num{3.12e+01} \pm $\\$ \num{3.50e-01}$} & $0.727$ & \makecell{$\num{1.50e+01} \pm $\\$ \num{2.24e-01}$} & \makecell{$\num{7.21e+00} \pm $\\$ \num{8.62e-02}$} & \makecell{$\num{1.71e+01} \pm $\\$ \num{1.97e-01}$} \\
        \hline
        $70$ & \makecell{$\num{4.08e+01} \pm $\\$ \num{6.12e-01}$} & \makecell{$\num{4.11e+01} \pm $\\$ \num{4.61e-01}$} & $0.671$ & \makecell{$\num{1.97e+01} \pm $\\$ \num{2.95e-01}$} & \makecell{$\num{9.42e+00} \pm $\\$ \num{1.10e-01}$} & \makecell{$\num{2.26e+01} \pm $\\$ \num{2.58e-01}$} \\
        \hline
        $80$ & \makecell{$\num{5.18e+01} \pm $\\$ \num{7.78e-01}$} & \makecell{$\num{5.22e+01} \pm $\\$ \num{5.79e-01}$} & $0.627$ & \makecell{$\num{2.49e+01} \pm $\\$ \num{3.74e-01}$} & \makecell{$\num{1.19e+01} \pm $\\$ \num{1.36e-01}$} & \makecell{$\num{2.86e+01} \pm $\\$ \num{3.24e-01}$} \\
        \hline
        $90$ & \makecell{$\num{6.40e+01} \pm $\\$ \num{9.60e-01}$} & \makecell{$\num{6.44e+01} \pm $\\$ \num{7.10e-01}$} & $0.600$ & \makecell{$\num{3.07e+01} \pm $\\$ \num{4.60e-01}$} & \makecell{$\num{1.45e+01} \pm $\\$ \num{1.65e-01}$} & \makecell{$\num{3.53e+01} \pm $\\$ \num{3.96e-01}$} \\
        \hline
        $100$ & \makecell{$\num{7.72e+01} \pm $\\$ \num{1.16e+00}$} & \makecell{$\num{7.76e+01} \pm $\\$ \num{8.57e-01}$} & $0.580$ & \makecell{$\num{3.70e+01} \pm $\\$ \num{5.54e-01}$} & \makecell{$\num{1.74e+01} \pm $\\$ \num{1.95e-01}$} & \makecell{$\num{4.26e+01} \pm $\\$ \num{4.72e-01}$} \\
        \hline
        $125$ & \makecell{$\num{1.15e+02} \pm $\\$ \num{1.72e+00}$} & \makecell{$\num{1.15e+02} \pm $\\$ \num{1.26e+00}$} & $0.536$ & \makecell{$\num{5.47e+01} \pm $\\$ \num{8.20e-01}$} & \makecell{$\num{2.56e+01} \pm $\\$ \num{2.81e-01}$} & \makecell{$\num{6.31e+01} \pm $\\$ \num{6.90e-01}$} \\
        \hline
        $150$ & \makecell{$\num{1.58e+02} \pm $\\$ \num{2.37e+00}$} & \makecell{$\num{1.59e+02} \pm $\\$ \num{1.69e+00}$} & $0.607$ & \makecell{$\num{7.51e+01} \pm $\\$ \num{1.13e+00}$} & \makecell{$\num{3.49e+01} \pm $\\$ \num{3.79e-01}$} & \makecell{$\num{8.68e+01} \pm $\\$ \num{9.35e-01}$} \\
        \hline
        $175$ & \makecell{$\num{2.06e+02} \pm $\\$ \num{3.09e+00}$} & \makecell{$\num{2.07e+02} \pm $\\$ \num{2.20e+00}$} & $0.569$ & \makecell{$\num{9.80e+01} \pm $\\$ \num{1.47e+00}$} & \makecell{$\num{4.52e+01} \pm $\\$ \num{4.87e-01}$} & \makecell{$\num{1.13e+02} \pm $\\$ \num{1.20e+00}$} \\
        \hline
        $200$ & \makecell{$\num{2.60e+02} \pm $\\$ \num{3.89e+00}$} & \makecell{$\num{2.61e+02} \pm $\\$ \num{2.74e+00}$} & $0.501$ & \makecell{$\num{1.23e+02} \pm $\\$ \num{1.85e+00}$} & \makecell{$\num{5.66e+01} \pm $\\$ \num{6.08e-01}$} & \makecell{$\num{1.43e+02} \pm $\\$ \num{1.51e+00}$} \\
        \hline
        $225$ & \makecell{$\num{3.17e+02} \pm $\\$ \num{4.76e+00}$} & \makecell{$\num{3.19e+02} \pm $\\$ \num{3.33e+00}$} & $0.500$ & \makecell{$\num{1.50e+02} \pm $\\$ \num{2.25e+00}$} & \makecell{$\num{6.89e+01} \pm $\\$ \num{7.37e-01}$} & \makecell{$\num{1.74e+02} \pm $\\$ \num{1.87e+00}$} \\
        \hline
        $250$ & \makecell{$\num{3.79e+02} \pm $\\$ \num{5.69e+00}$} & \makecell{$\num{3.81e+02} \pm $\\$ \num{3.98e+00}$} & $0.490$ & \makecell{$\num{1.79e+02} \pm $\\$ \num{2.69e+00}$} & \makecell{$\num{8.20e+01} \pm $\\$ \num{8.68e-01}$} & \makecell{$\num{2.08e+02} \pm $\\$ \num{2.29e+00}$} \\
    \end{tabular}\label{tab:wet:measurements}
\end{table}


\section{Medical Paper/Heterogeneous Phantom}
Lastly, a measurement with a heterogeneous LN300 Gammex phantom was done, shown in Figure~\ref{fig:detector:measurement:hetero}.
The phantom has a thickness of \SI{200}{\mm}, a physical density of \SI{0.30}{\gram\per\centi\meter^3} and an electron density relative to water of $0.290$~\citep{gammexphantom}.
The LN300 phantom has a modulation power of \SI{210}{\micro\meter}, which was measured using a PTW PeakFinder.

\begin{figure}[htb]
    \centering
    \includegraphics[width=.5\textwidth]{figs/heterogeneousMeasure.jpg}
    \caption{Heterogeneous phantom setup: \SI{200}{\mm} Gammex LN300 lung equivalent phantom.}\label{fig:detector:measurement:hetero}
\end{figure}

Two approaches are presented to analyze the depth dose spectra and extract the modulation power.
The first is the standard convolution of a linear interpolated reference curve and fitting the result to the depth dose spectra of the heterogeneous phantom~\citep{Baumann_2017}.
The second approach uses the analytical approximation of the depth dose curve from~\citet{bortfeld}, given in Equation~\ref{eq:depthdose}.
For this we will shown that the modulation powers parameters are indirectly accessible by Equation~\ref{eq:depthdose} without having to do an extra convolution.

\begin{align}
  %   \hat{D}(z) &=
  % \begin{cases}
  %     \Phi_0 \frac{(R_0-z)^{1/p-1}+(\beta+\gamma\beta p)(R_0-z)^{1/p}}{\rho p \alpha^{1/p} (1+\beta R_0)}, & \text{if } z < R_0 \\
  %     0, & \text{if } z>R_0
  % \end{cases} \\ \label{eq:depthdose:nostraggling}
  D(z) &= (\hat{D} \ast \mathcal{N})(z) = \frac{1}{\sqrt{2\pi}\sigma}\int^{R_0}_{-\infty} \hat{D}(\overline{z}) e^{-\frac{(z-\overline{z})^2}{2\sigma^2}}d\overline{z}\label{eq:depthdose:straggling} \\ 
  &= \Phi_0 \frac{e^{-\zeta^2/4}\sigma^{1/p}\Gamma(1/p)}{\sqrt{2\pi}\rho p \alpha^{1/p}(1+\beta R_0)}\cdot \left[\frac{1}{\sigma}\mathcal{D}_{-1/p}(-\zeta)+\left(\frac{\beta}{p}+\gamma\beta+\frac{\epsilon}{R_0}\right)\mathcal{D}_{-1/p-1}(-\zeta) \right],~\zeta=\frac{R_0-z}{\sigma} \label{eq:depthdose:hetero}
\end{align}

We start from the pre-convolved depth dose distribution, shown in Equation~\ref{eq:depthdose:straggling}, which is given as the depth dose distribution without range straggling $\hat{D}$ convolution with a range staggling normal distribution $\mathcal{N}$.
The depth dose distribution after passing a heterogeneous phantom $D_h(z; t, \sigma_t)$ is calculated by performing an second convolution with a normal distribution $\mathcal{N}_t$, where the displacement $t$ and broadening $\sigma_t$ are the parameters of the modulation power.
Using the associativity of convolutions, we first convolve the two normal distributions, giving a single normal distribution $\mathcal{N}_h$ with standard deviation $\sigma_h=\sqrt{\sigma^2+\sigma_t^2}$ and mean $\mu_h=z-t$.
From there, we shift the integral limit by $t$ and change the fluence normalization to account for the changed range $R_t=R_0-t$ by defining $\Phi_0'$ as the fluence after the heterogeneous target with $\frac{1+\beta (R_0-t)}{1+\beta R_0}$ being the fluence reduction factor.
This results in the final depth dose distribution for the heterogeneous phantom shown in Equation~\ref{eq:depthdose:heterogeneous}, which has the same form as the depth dose distribution without a target except that the range $R_0$ is reduced by the displacement $t$, the width $\sigma$ is broadened according to the quadrature sum of the two individual widths and the fluence $\Phi_0$ is adjusted by the lost particles in the phantom to $\Phi_0'$.

\begin{align}  
  D_h(z; t, \sigma_t) &= (\hat{D} \ast \mathcal{N} \ast \mathcal{N}_t)(z) = (\hat{D} \ast \mathcal{N}_h)(z)\\
  &= \frac{1}{\sqrt{2\pi}\sigma_h}\int^{R_0}_{-\infty} \hat{D}(\overline{z}) e^{-\frac{(z-t-\overline{z})^2}{2\sigma_h^2}}d\overline{z}\\
  &= \frac{1}{\sqrt{2\pi}\sigma_h}\int^{R_0-t}_{-\infty} \hat{D}(\tilde{z}-t) e^{-\frac{(z-\tilde{z})^2}{2\sigma_h^2}}d\tilde{z}\\
  &= \Phi_0' \frac{e^{-\zeta'^2/4}\sigma_h^{1/p}\Gamma(1/p)}{\sqrt{2\pi}\rho p \alpha^{1/p}(1+\beta R_t)}\cdot \left[\frac{1}{\sigma_h}\mathcal{D}_{-1/p}(-\zeta')+\left(\frac{\beta}{p}+\gamma\beta+\frac{\epsilon}{R_t}\right)\mathcal{D}_{-1/p-1}(-\zeta') \right],~\zeta'=\frac{R_t-z}{\sigma_h}\\
  &= D \Big(z; R_t = R_0 - t, \sigma_h = \sqrt{\sigma^2+\sigma_t^2}, \Phi_0' = \Phi_0\frac{1+\beta (R_0-t)}{1+\beta R_0}\Big)\label{eq:depthdose:heterogeneous}
\end{align}

\begin{figure}[htb]
  \centering
  \includegraphics[width=0.6\textwidth]{figs/braggfitHetero.pdf}
  \caption{Depth--dose measurements and fits of measurements without a phantom and with a heterogeneous LN300 phantom.}\label{fig:ddd:fitted:hetero}
\end{figure}


The mean nominal LN300 phantom thickness $t_{\mathrm{nominal}}$ can be calculated from the electron density ratio to water of $0.28$ to \SI{5.8}{\cm}.
The measured value of $5.836\pm 0.018\,\si{\cm}$ agrees with this within \SI{0.56}{\percent}, corresponding to an absolute deviation of \SI{360}{\micro\meter}.
The measured modulation power agrees with the nominal value within \SI{2.64}{\percent}, corresponding to an absolute deviation of \SI{5.541}{\micro\meter}.

\begin{table}[htb]
    \centering
    \captionof{table}{Fit results obtained from depth--dose distributions measured without a phantom and with the heterogeneous LN300 phantoms. From left to right: the range of the distal fall-off $R_0$, the width of the Bragg peak or range and energy straggling $\sigma$, the reconstructed WET of the LN300 phantom from range difference to the measurement without a target~$t$, the modulation Power $P_{\mathrm{mod}}$, the nominal $P_{\mathrm{mod, nom.}}$, the absolute deviation $\Delta P = P_{\mathrm{mod}} - P_{\mathrm{mod, nom.}}$ and the relative deviation $\Delta P_{\%}$. The deviations are given with respect to the nominal modulation power $P_{\mathrm{mod, nom.}}$, measured using a PTW peakfinder.}\label{tab:ddd:fitted:hetero}
    \sisetup{
      separate-uncertainty,
      table-number-alignment = center
    }
    \begin{tabular}{lcccccccc}
    \toprule
    Phantom & $R_0$ / cm  & $\sigma$ / cm & $t$ / cm & $P_{\mathrm{mod}}$ / \si{\micro\meter} & $P_{\mathrm{mod, nom.}}$ / \si{\micro\meter}  &$\Delta t$ / \si{\micro\meter} & $\Delta t_{\%}$ / \% \\
    \midrule
    None & $30.991 \pm 0.002$ & $0.489 \pm 0.004$ & - & - & - & - & - \\
    LN300 & $25.156 \pm 0.018$ & $0.604\pm 0.028$ & $5.836\pm 0.018$ &  $215.541 \pm 9.886$ & $210$ & 5.541 & $2.64$ \\
    \bottomrule
    \end{tabular}
\end{table}


\clearpage
\pagenumbering{Roman}
% Appendix
%\input{appendix.tex}

\printbibliography[heading=bibintoc]

\listoffigures
\listoftables
\printglossary[style=mymcolalttree, type=\acronymtype,title={List of Abbreviations}, toctitle={List of Abbreviations}]

\includepdf[pages=-]{Selbststaendigkeitserklaerung.pdf}
\end{document}